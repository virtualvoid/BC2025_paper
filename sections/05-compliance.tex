\section{Legislative Compliance Model}

The records management component of the system is designed to reflect established principles of structured classification and lifecycle control. 
The data model incorporates hierarchical classification entities that represent subject-based groupings of records. 
These classification structures allow consistent assignment of documents to predefined categories and support standardized numbering schemes.

The core domain model includes entities representing evidence types, volumes, records, and case files. 
This structure enables separation between abstract record categories and their concrete yearly or contextual instances. 
Such modeling supports scalable growth of registry data while preserving logical grouping and traceability across different operational contexts.

Retention and disposal mechanisms are incorporated into the data model as lifecycle attributes associated with classification entities. 
While the current implementation provides structural support for defining retention periods and disposal types, the lifecycle automation layer is designed for further extensibility. 
This approach allows gradual refinement of compliance logic without altering the core data schema.

To ensure traceability and accountability, the system maintains audit-related metadata across key entities. 
Creation and modification timestamps, as well as user attribution fields, are persistently stored and integrated with the authentication subsystem. 
This design supports transparency and aligns with compliance-oriented requirements for records management systems.

By integrating hierarchical classification, lifecycle attributes, and auditability into the core domain model, 
the system establishes a structured foundation for regulatory compliance while remaining adaptable to evolving organizational requirements.


The compliance model is derived from Slovak archival legislation, including Act No. 395/2002 Coll. on Archives and Registries \cite{zakon395} and Decree No. 628/2002 Coll. \cite{vyhlaska628}, while also considering EU-level regulations such as eIDAS \cite{eidas} and GDPR \cite{gdpr2016}.
