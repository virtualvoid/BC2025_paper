\section{System Architecture}

The proposed system follows a modular service-oriented architecture designed to ensure scalability, extensibility, and separation of concerns. 
The backend is implemented using the .NET 10 platform \cite{microsoft_dotnet} and is structured into multiple logical components responsible for records management, workflow execution, authentication, and supporting infrastructure services.

Structured metadata related to records, classification hierarchies, workflow definitions, and execution states are stored in a relational database (Microsoft SQL Server). 
This choice enables strong consistency guarantees and transactional integrity for compliance-sensitive operations.

Binary document content is stored separately from structured metadata. 
Uploaded files are persisted in a document-oriented storage system (MongoDB), which is optimized for handling large binary objects and scalable file storage. 
This architectural separation between metadata and file content improves flexibility, supports horizontal scaling of storage components, and reduces performance bottlenecks in relational persistence layers.

Inter-service communication and asynchronous processing are supported by a message broker (RabbitMQ), which facilitates decoupled execution of background tasks and workflow-related events. 
A distributed caching layer (Redis) is employed to improve performance for frequently accessed metadata and configuration data.

Authentication and authorization are handled via an external identity provider (Keycloak), enabling standards-based authentication protocols and role-based access control. 
The system exposes its functionality through a REST-based API layer, allowing integration with external systems and frontend clients.

The user interface is implemented as a single-page application using React and TypeScript. 
A component-based UI framework is employed to ensure consistency and maintainability. 
This separation between backend services and frontend client supports independent evolution of presentation and business logic layers.

For deployment, the system is containerized and supports Docker-based orchestration. 
The architecture is designed to be environment-agnostic, enabling both local development deployments and cloud-ready configurations. 
This infrastructure approach aligns with modern enterprise deployment practices and ensures portability across different hosting environments.


The architectural decisions follow established enterprise design principles \cite{fowler,sommerville} and align with modern service-oriented system patterns.

\subsection{Configurable Metadata Model}

In addition to core record attributes, the system supports configurable metadata bound to the evidence type (registry evidence category). 
Each evidence type defines a set of user-defined fields (metadata schema) that are automatically projected into the record creation and update forms. 
This approach allows organizations to adapt the system to heterogeneous document agendas without code changes, while preserving a unified storage and audit model.

Metadata fields are stored as structured key--value pairs with explicit typing and validation rules (e.g., text, numeric, boolean, date, or enumerations). 
At runtime, the frontend dynamically renders form controls based on the evidence-level metadata definition, and the backend enforces validation to prevent schema drift and ensure data quality.

From a compliance perspective, evidence-bound metadata improves traceability and classification quality by ensuring that domain-specific attributes required by internal policies or legislation are consistently captured during record registration. 
This aligns with standard records management principles that emphasize controlled metadata, retention context, and reproducibility of record descriptions \cite{iso15489,moreq2010}.

The design follows a schema-driven approach, reducing implementation-level customization and enabling controlled evolution of metadata requirements through configuration rather than redeployment.
