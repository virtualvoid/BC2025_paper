\section{System Architecture}

The proposed system follows a modular service-oriented architecture designed to ensure scalability, extensibility, and separation of concerns. 
The backend is implemented using the .NET 10 platform \cite{microsoft_dotnet} and is structured into multiple logical components responsible for records management, workflow execution, authentication, and supporting infrastructure services.

Structured metadata related to records, classification hierarchies, workflow definitions, and execution states are stored in a relational database (Microsoft SQL Server). 
This choice enables strong consistency guarantees and transactional integrity for compliance-sensitive operations.

Binary document content is stored separately from structured metadata. 
Uploaded files are persisted in a document-oriented storage system (MongoDB), which is optimized for handling large binary objects and scalable file storage. 
This architectural separation between metadata and file content improves flexibility, supports horizontal scaling of storage components, and reduces performance bottlenecks in relational persistence layers.

Inter-service communication and asynchronous processing are supported by a message broker (RabbitMQ), which facilitates decoupled execution of background tasks and workflow-related events. 
A distributed caching layer (Redis) is employed to improve performance for frequently accessed metadata and configuration data.

Authentication and authorization are handled via an external identity provider (Keycloak), enabling standards-based authentication protocols and role-based access control. 
The system exposes its functionality through a REST-based API layer, allowing integration with external systems and frontend clients.

The user interface is implemented as a single-page application using React and TypeScript. 
A component-based UI framework is employed to ensure consistency and maintainability. 
This separation between backend services and frontend client supports independent evolution of presentation and business logic layers.

For deployment, the system is containerized and supports Docker-based orchestration. 
The architecture is designed to be environment-agnostic, enabling both local development deployments and cloud-ready configurations. 
This infrastructure approach aligns with modern enterprise deployment practices and ensures portability across different hosting environments.


The architectural decisions follow established enterprise design principles \cite{fowler,sommerville} and align with modern service-oriented system patterns.
