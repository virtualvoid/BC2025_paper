\section{Discussion}

The presented system demonstrates that integrating records management and workflow execution within a unified architecture can significantly reduce architectural fragmentation typically observed in enterprise environments. 
By avoiding separation between document storage and process orchestration layers, the system maintains consistent state management and minimizes synchronization complexity.

The service-oriented backend architecture enables horizontal scalability and modular evolution of individual components. 
The use of asynchronous messaging for workflow-related events decouples execution logic from synchronous API interactions, improving responsiveness and resilience under load. 
The separation between workflow definitions and runtime instances further enhances maintainability, as structural changes in process models do not affect already running workflow executions.

From a performance perspective, the introduction of a distributed caching layer reduces database load for frequently accessed configuration and metadata entities. 
The reliance on a relational database for compliance-critical operations ensures transactional integrity and predictable behavior, which is particularly important in records lifecycle management scenarios.

Despite these advantages, certain limitations remain. While the system provides structural support for retention policies and disposal types, 
automated lifecycle enforcement mechanisms require further refinement and real-world validation. Additionally, large-scale deployment scenarios would benefit from extended performance benchmarking and distributed execution analysis.

Nevertheless, the architectural design establishes a solid technical foundation for enterprise-grade document and records management systems, combining configurability, traceability, and scalability within a cohesive framework.
