\section{Workflow Engine Design}

The workflow engine represents a central architectural component of the proposed system. 
Its purpose is to integrate process orchestration directly into the records management platform, 
thereby eliminating the need for external workflow subsystems and reducing integration complexity. 
The engine is designed to support configurability, version control, persistent execution state, 
and dynamic evaluation of process logic.

Unlike traditional hard-coded business logic implementations, the workflow layer allows process models 
to be defined, modified, and versioned independently of application redeployment. 
This design enables organizations to adapt their document-related processes 
(e.g., approvals, registrations, reviews, or disposal procedures) 
without modifying the core application codebase.

\subsection{Workflow Definition Model}

Workflow processes are modeled as directed graphs consisting of nodes (steps) and transitions. 
The definition layer includes explicit support for versioning: each workflow definition is stored together with a snapshot of its structure, 
and newer versions maintain a reference to the previous version. 
This approach allows controlled evolution of workflow logic while preserving historical definitions for traceability and reproducibility.

The engine supports multiple node types, including start and end nodes, human task nodes, decision nodes, and script nodes. 
Transitions define the control flow between nodes and may include conditions or metadata used by the execution layer.

\subsection{Execution Model}

Workflow execution is represented by workflow instances that reference a specific version of a workflow definition. 
Each instance maintains persistent execution state, including the current node, completed steps, and contextual data required to evaluate transitions and decision logic. 
Persisting execution state enables fault tolerance and supports auditing of workflow progress over time.

Asynchronous execution and background processing are supported through message-based task dispatching, 
allowing workflow-related activities to be decoupled from synchronous API calls. 
This model improves scalability and helps ensure that long-running workflow steps do not block user interactions.

\subsection{Script Nodes and Dynamic Logic}

To enable flexible business logic without redeployment, the workflow engine provides script nodes. 
These nodes execute embedded scripts during runtime to evaluate conditions, compute values, 
or perform lightweight validation and routing logic. Script node outputs can influence transition selection 
and enable dynamic behavior in workflow graphs while keeping the core execution engine stable.

Overall, the workflow engine design balances configurability and maintainability by separating workflow structure (definitions) from runtime state (executions), 
while ensuring that versioned definitions remain available for auditing and consistent replay of historical workflow instances.
