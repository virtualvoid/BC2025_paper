\section{Background and Related Work}
Document and records management systems have been extensively studied in the context of enterprise information systems. 
Standards such as MoReq and ISO 15489 define principles for records lifecycle management, including classification, retention policies, disposal procedures, and auditability. 
These frameworks emphasize the importance of structured metadata, hierarchical classification schemes, and controlled records disposition processes.

Modern enterprise software increasingly adopts service-oriented or microservice-based architectures to improve scalability and modularity. 
In parallel, workflow management systems have evolved to support dynamic process modeling, configurable execution logic, and distributed task orchestration. 
However, many existing solutions treat document management and workflow orchestration as separate subsystems, connected through integration layers.

This separation can lead to inconsistencies in state management, duplication of lifecycle information, and increased architectural complexity. 
Several research works propose tighter integration between content management and process execution, yet practical implementations often prioritize either document storage or workflow flexibility.

The solution presented in this paper builds upon established records management principles while integrating a configurable workflow engine directly into the core system architecture. 
The objective is not to redefine records management standards, but to demonstrate how compliance-oriented data modeling and workflow execution can coexist within a unified enterprise system.
